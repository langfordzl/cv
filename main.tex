%----------------------------------------------------------------------------------------
%	PACKAGES AND OTHER DOCUMENT CONFIGURATIONS
%----------------------------------------------------------------------------------------

\documentclass{tidycv} % Use the custom tidycv.cls style

\newcommand{\tab}[1]{\hspace{.2667\textwidth}\rlap{#1}}
\newcommand{\itab}[1]{\hspace{0em}\rlap{#1}}

\name{Zachary}{Langford}
\position{R\&D Associate, Cyber \& Applied Data Analytics Division}
\address{Cyber Physical Systems Group, Oak Ridge National Laboratory}
% \mobile{(+00) 00-0000-0000}
\email{zlangfor@vols.utk.edu}
\homepage{langfordzl.github.io}
\github{langfordzl}
\linkedin{zachary-langford-91857b164}
% \gitlab{gitlab-id}
% \stackoverflow{SO-id}{SO-name}
% \twitter{@twit}
% \skype{skype-id}
% \reddit{reddit-id}
% \extrainfo{extra informations}

\makeatletter
%\renewcommand\@biblabel[1]{aaa}
\renewenvironment{thebibliography}[1]
     {\sectionskip
      \vspace{-.7em}
      \@mkboth{\MakeUppercase\refname}{\MakeUppercase\refname}%
      \list{}%
           {\leftmargin0pt
            \@openbib@code
            \usecounter{enumiv}}%
      \sloppy
      \clubpenalty4000
      \@clubpenalty \clubpenalty
      \widowpenalty4000%
      \sfcode`\.\@m}
     {\def\@noitemerr
       {\@latex@warning{Empty `thebibliography' environment}}%
      \endlist}
\makeatother

\begin{document}

\makecvheader


%----------------------------------------------------------------------------------------
%	Research Interest
%----------------------------------------------------------------------------------------
\cvsection{Overview}
\cvresearchinterest{
I completed my PhD in Energy Science and Engineering from the University of Tennessee. 
My PhD research applied data mining methods on remote sensing imagery and environmental datasets for landscape characterization in Arctic ecosystems. I received my M.Sc. in Civil Engineering from Penn State University focusing on object-based imagery analysis and artificial neural networks.
I recently worked for Virginia Tech, focusing on developing data analytic frameworks for national security projects, and Boeing Research \& Technology, developing geospatial and machine learning tools for the aerospace industry.  
%My current research interests include data mining applications in scientific disciplines such as cyber-physical systems, image/signal processing, geospatial modeling, and environmental sciences.
}


%----------------------------------------------------------------------------------------
%	Research Interest
%----------------------------------------------------------------------------------------
\cvsection{Research Interest}
\cvresearchinterest{Data analytics in engineering systems, environmental modeling, spatial statistics, reinforcement learning, meta-learning, environmental remote sensing, geospatial modeling}

%----------------------------------------------------------------------------------------
%	EDUCATION
%----------------------------------------------------------------------------------------

\cvsection{Education}
\begin{cveducations}
  
  \cveducation{
    \begin{cveducationitems2}
         2017 \\ 
	\end{cveducationitems2}
    } % Year
    {{\bf University of Tennessee}} % Institution
    {
    \begin{cveducationitems} % Description(s)
        PhD in Energy Science \& Engineering \hfill Advisor: Dr. Jitendra Kumar (ORNL)\\ Dissertation: Spatiotemporal  Characterization of Arctic Landscapes Using Geospatial Analytics\\
	\end{cveducationitems}
    }
    
  \cveducation{2013} % Year
    {{\bf Pennsylvania State University -- University Park}} % Institution
    {
    \begin{cveducationitems} % Description(s)
        MS in Civil Engineering \hfill Advisor: Dr. Michael Gooseff (CU-Boulder) \\ Concentration: Remote Sensing, Machine Learning\\
	\end{cveducationitems}
    }
    
  \cveducation{2011} % Year
    {{\bf University of Alabama -- Huntsville}} % Institution
    {
    \begin{cveducationitems} % Description(s)
        BS in Earth System Science \\Minor: Mathematics\\
	\end{cveducationitems}
    }    
    
%      \cveducation{2020--Current} % Year
%    {{\bf University of Tennessee}} % Institution
%    {
%    \begin{cveducationitems} % Description(s)
%        Graduate Certificate
%	\end{cveducationitems}
%    }   
    
\end{cveducations}


%-------------------------------------------------------------------------------
%	RESEARCH EXPERIENCE
%-------------------------------------------------------------------------------
%% usage: \begin{cvresearch}{<Project title>}{<Date>}{<Position>}{<Institution>}
%%			\item <Description>
%%			\item <Description>
%%		  \end{cvresearch}
\cvsection{Research Experience}
\begin{cvresearches}
%------------------------------------------------
\begin{cvresearch}{R{\&}D Associate}{Aug. 2019 -- present}{Cyber \& Applied Data Analytics Division}{Oak Ridge National Laboratory}
 \item Developing data analytics techniques (e.g. anomaly detection, visualizations, machine learning, etc.) for energy security.
 \item Machine learning algorithms for developing digital twins of programming logic controllers (PLCs).
 \item Contributing to research proposals/papers around machine learning and data mining.
 \item Data augmentation for wireless signal classification. 
\end{cvresearch}
%------------------------------------------------
\begin{cvresearch}{Research Scientist}{Jan. 2019 -- Aug. 2019}{Hume Center for National Security \& Technology}{Virginia Tech}
 \item Developed new approaches for signal classification using deep learning methods. 
 \item Developed siamese networks for better classification of signals with high signal-to-noise ratio.
 \item Generative Adversarial Network modeling of latent space for extracting wanted features of synthetic images. 
 \item Contributed to Bayesian networks for understanding the likelihoods of known causes and contributing factors. 
% \item Publication: \cite{Langford:2019:RSC:3324921.3328781}
\end{cvresearch}

\pagebreak
%------------------------------------------------
\begin{cvresearch}{Advanced Technologist}{Aug. 2016 -- Jan. 2019}{Boeing Research \& Technology}{Boeing}
 \item Geospatial machine learning applications for synthetic modeling of environments.
 \item Hyperspectral remote sensing sensor modeling (using DIRSIG) for methane detection.
 \item Multi/Hyperspectral image classification using deep learning techniques. 
 \item Geospatial modeling (raster, vector, LiDAR processing) for inputs into autonomous aircraft taxing. 
\end{cvresearch}

\end{cvresearches}


%------------------------------------------------
\cvsection{Graduate Research Experience}
\begin{cvresearches}

\begin{cvresearch}{Graduate Research Assistant}{Aug. 2014 -- Aug. 2016}{Bredesen Center}{University of Tennessee}
 \item Working with Dr. Jitendra Kumar and Dr. Forrest Hoffman in the Climate Change Science Institute at ORNL.
 \item Provided geospatial datasets of Arctic ecosystems for parameterizing land surface models for climate change research.
 \item  Developed multi-sensor fusion frameworks of satellite imagery and ground measurements for retrieving surface parameters.
 \item  Developed automated algorithms for identifying and understanding disturbance threats
(e.g., wildfires) using spatiotemporal datasets in Google Earth Engine.
\end{cvresearch}

\begin{cvresearch}{ASTRO Internship}{Jan. 2014 -- Aug. 2014}{Computational Earth Sciences Group}{Oak Ridge National Laboratory}
 \item Program offering short-term research opportunities with ORNL to current/recent graduate students. 
 \item Developed geospatial algorithms on high-resolution satellite imagery (i.e. 20 cm) for upscaling ground-based measurements.
 \item Continued research while enrolling into UTK/ORNL PhD program.
\end{cvresearch}

\begin{cvresearch}{Graduate Research Assistant}{2011 -- 2013}{Department of Civil and Environmental Engineering}{Pennsylvania State University}
 \item Utilized high-resolution, 50 cm -- 2 m, panchromatic \& multispectral satellite imagery
(QuickBird, WorldView-2, and GeoEye) to characterize soil moisture profiles in the McMurdo Dry Valleys, Antarctica using an object-based image classification. 
 \item Create a feed-forward neural network to estimate soil moisture values using multispectral satellite imagery and ground-based measurements.
\end{cvresearch}

\begin{cvresearch}{Undergraduate Research Assistant}{2010 -- 2011}{Department of Atmospheric Sciences}{University of Alabama in Huntsville}
\item Collaborated with NASA's Regional Visualization and Monitoring System (SERVIR) and
Smithsonian Tropical Research Institute (STRI) to integrate satellite and geospatial data
to support environmental monitoring in Central America.
\end{cvresearch}

\end{cvresearches}


%----------------------------------------------------------------------------------------
%	HONORS AND AWARDS
%----------------------------------------------------------------------------------------

\cvsection{Honors and Services}
\begin{cvhonors}
  %\cvhonor{2018}{{\bf Scamander Scholarship}}{
  %	\begin{cvhonoritems} % Description(s)
 %       full tuition and stipend for 12 months and research-related expenses (total $\$75,200$)\\
%	\end{cvhonoritems}}
  \cvhonor{2014--2017}{Energy Science \& Engineering PhD Fellowship}{ORNL/UTK}
  \cvhonor{2014 -- Present}{Reviewer}{
  	\begin{cvhonoritems} % Description(s)
        Energies, Journal of Spatial Science  \hfill 2020\\
        Symmetry, Geophysical Research Letters \hfill 2019\\
        IEEE ICDM Workshop on
Data Mining in Earth System Science \hfill 2017--2019\\
        Remote Sensing \hfill 2014--2020\\
	\end{cvhonoritems}}
\end{cvhonors}


\cvsection{Research Proposals}

\begin{cvhonors}

  \cvhonor{2020 (Submitted)}{(co-PI) FAIR Data and Models for Artificial Intelligence and Machine Learning}{U.S. Department of Energy (DOE)}

\end{cvhonors}

\pagebreak
%----------------------------------------------------------------------------------------
% TECHNICAL STRENGTHS
%----------------------------------------------------------------------------------------
%% usage: \cvtechnicalstrength{<Type>}{<List>} 
\cvsection{Technical Strengths}
\begin{cvtechnicalstrengths}
  \cvtechnicalstrength
  {Languages} % Type
  {Python (Advanced), MATLAB (Proficient), C/C++ (Proficient), CUDA (Beginner), MPI (Beginner) } % List
  \cvtechnicalstrength
  {Python Libraries} % Type
  {TensorFlow, PyTorch, MXNet, Scikit-learn, GeoPandas} % List
   \cvtechnicalstrength
  {GIS/Remote Sensing Tools} % Type
  {GRASS GIS, QGIS, ArcGIS, Google Earth Engine} % List 
   \cvtechnicalstrength
  {Other Tools} % Type
  {LaTeX, SQL, Elasticsearch, DIRSIG} % List 
\end{cvtechnicalstrengths}

%----------------------------------------------------------------------------------------
%	PUBLICATION
%----------------------------------------------------------------------------------------


\cvsection{Publications}
\begin{cvpublications}
\begin{bibunit}[tidycv]
  \nocite{Langford:2019:RSC:3324921.3328781}
  \nocite{Langford_DMESS2019_20191108}
  \nocite{Langford_RemoteSens_20190102}
  \nocite{Langford_DMESS2018_20181117}
   \nocite{Langford_DMESS2017_20171118}
   \nocite{Langford_RemoteSens_20160906}
   \nocite{Langford_2015}
  \putbib[publication_public]
\end{bibunit}
\end{cvpublications}


\cvsection{Conference Abstracts}
%% usage 1: insert publications using bibentry
\begin{cvpublications}
\begin{bibunit}[tidycv]

%2019
 \nocite{AGU_2019}
  \nocite{ Hoffman_IALE_2019}
  
 %2018
 \nocite{Langford_AGU_2018}
% \nocite{Langford_ICDM_2018}
 \nocite{Langford_ICEI_2018}
 \nocite{Langford_FOSS4G_2018}
 \nocite{Langford_IALE_2018}
 \nocite{Hoffman_DOE_2018}
 
 %2017
 \nocite{Langford_ICDM_2017}
 
 %2016
 \nocite{Langford_IALE_2016}
 \nocite{Hoffman_ESS}
 \nocite{Langford_AGU_2015}
 \nocite{Kumar_AGU_2015}
 
 \nocite{Hoffman_NGEE_Arctic}
 \nocite{Hoffman_NGEE_Tropic}
 \nocite{Hoffman_ATBC}
 \nocite{Hoffman_IALE}
 \nocite{Hoffman_CSDMS}
 \nocite{Hoffman_AGU}
 \nocite{Langford_AGU_2014}
 \nocite{Kumar_AGU_2014}
 \nocite{Hoffman_DOE_Amazon}
 \nocite{Hoffman_DOE}
 
 %2012
 \nocite{Langford_AGU_2012}
 \nocite{Langford_LTER}
 \nocite{Langford_SCAR}
 
 %2011
  \nocite{Langford_IAC_2011_1}
  \nocite{Langford_IAC_2011_2}
  \putbib[publication_public]
\end{bibunit}
\end{cvpublications}


\cvsection{Datasets}
%% usage 1: insert publications using bibentry
\begin{cvpublications}
\begin{bibunit}[tidycv]
  \nocite{osti_1123668}
  \nocite{osti_1418854}
  \putbib[publication_public]
\end{bibunit}
\end{cvpublications}

\pagebreak
\cvsection{Theses and Dissertations}
%% usage 1: insert publications using bibentry
\begin{cvpublications}
\begin{bibunit}[tidycv]
 \nocite{Langford-Dissertation_2017}
  \nocite{Langford-Theses_2013}
  \putbib[publication_public]
\end{bibunit}
\end{cvpublications}

\cvsection{Patents}
%% usage 1: insert publications using bibentry
\begin{cvpublications}
\begin{bibunit}[tidycv]
  \nocite{langford_20190158805}
  \putbib[publication_public]
\end{bibunit}
\end{cvpublications}



%-------------------------------------------------------------------------------
% SOFTWARE
%-------------------------------------------------------------------------------
%\cvsection{Software}
%\begin{cvsoftwares}
%  \cvsoftware
%  {LaTeX template} % Type
%  {{\bf TidyCV}: simple and tidy LaTeX template for your curriculum vitae} % Title
% {https://github.com/SeojinBang/TidyCV} % URL
%\end{cvsoftwares}




%-------------------------------------------------------------------------------
% PROFESSIONAL SERVICES
%-------------------------------------------------------------------------------
%\cvsection{Professional Service}
%\begin{cvprofessionals}
%  \cvprofessional{2020}{Reviewer}{Energies, Remote Sensing, Journal of Spatial Science, Geophysical Research Letters}{}
%  \cvprofessional{2019}{Reviewer}{Symmetry, Remote Sensing, Geophysical Research Letters}{}
%   \cvprofessional{2018}{Reviewer}{Remote Sensing, Sensors, Algorithms}{}
%   \cvprofessional{2017-18}{Reviewer}{IEEE ICDM}{}
%   \cvprofessional{2014-16}{Reviewer}{Remote Sensing}{}
%\end{cvprofessionals}

%-------------------------------------------------------------------------------
% ATTENDED WORKSHOPS
%-------------------------------------------------------------------------------
\cvsection{Attended Workshops}
\begin{cvprofessionals}
  \cvprofessional{May 2020}{Shipley Associates}{Shipley's Proposal Writing Course}{}
  \cvprofessional{Nov. 2018}{IEEE ICDM Tutorial}{Summarizing Graphs at Multiple Scales}{}
  \cvprofessional{Jun. 2016}{University of Alaska Fairbanks}{ International Arctic Research Center (IARC) Summer School}{}
  \cvprofessional{Jul. 2014}{University of Minnesota}{Polar Geospatial Center Boot Camp}{} 
  \cvprofessional{2013}{Texas A{\&}M University}{Soil \& Water Assessment Tool Workshop}{}

\end{cvprofessionals}
%----------------------------------------------------------------------------------------
%	TEACHING EXPERIENCE
%----------------------------------------------------------------------------------------
%% usage: 
%% \cvsubsection{<Institution>}{<Position>}
%% \cvteaching{<Date>}{<Course title>}
%%\cvsection{Teaching Experience}
%\cvsubsection{{\bf Hogwarts School of Witchcraft and Wizardry}}{Teaching Assistant}
%\begin{cvteachings}
%\cvteaching
%  	{2018} % Year
%    {Concealment and Disguise} % Course Title
%  \cvteaching
%    {2017} % Year
 %   {Transfiguration} % Course Title
%\end{cvteachings}
%--------------------
%\cvsubsection{\bf Seoul National Magic School}{Teaching Assistant}
%\begin{cvteachings}
%  \cvteaching{2014}{Stealth and Tracking}
%  \cvteaching{2014}{Introduction to Magic I and II}
%\end{cvteachings}

%----------------------------------------------------------------------------------------
%	Publication at the End
%----------------------------------------------------------------------------------------


\end{document}
