%----------------------------------------------------------------------------------------
%	PACKAGES AND OTHER DOCUMENT CONFIGURATIONS
%----------------------------------------------------------------------------------------

\documentclass{tidycv} % Use the custom tidycv.cls style

\newcommand{\tab}[1]{\hspace{.2667\textwidth}\rlap{#1}}
\newcommand{\itab}[1]{\hspace{0em}\rlap{#1}}

\name{Zachary}{Langford}
\position{R\&D Associate, Cyber \& Applied Data Analytics Division}
\address{Oak Ridge National Laboratory \\ US Citizen, Q/SCI Clearance}
% \mobile{(+00) 00-0000-0000}
\email{zlangfor@vols.utk.edu}
\homepage{langfordzl.github.io}
\github{langfordzl}
\linkedin{zachary-langford-91857b164}
% \gitlab{gitlab-id}
% \stackoverflow{SO-id}{SO-name}
% \twitter{@twit}
% \skype{skype-id}
% \reddit{reddit-id}
% \extrainfo{extra informations}

\makeatletter
%\renewcommand\@biblabel[1]{aaa}
\renewenvironment{thebibliography}[1]
     {\sectionskip
      \vspace{-.7em}
      \@mkboth{\MakeUppercase\refname}{\MakeUppercase\refname}%
      \list{}%
           {\leftmargin0pt
            \@openbib@code
            \usecounter{enumiv}}%
      \sloppy
      \clubpenalty4000
      \@clubpenalty \clubpenalty
      \widowpenalty4000%
      \sfcode`\.\@m}
     {\def\@noitemerr
       {\@latex@warning{Empty `thebibliography' environment}}%
      \endlist}
\makeatother

\begin{document}

\makecvheader


%----------------------------------------------------------------------------------------
%	Research Interest
%----------------------------------------------------------------------------------------
%\cvsection{Overview}
%\cvresearchinterest{
%I completed my PhD in Energy Science and Engineering from the University of Tennessee, a joint program with Oak Ridge National Laboratory (ORNL). My PhD leveraged novel 
%machine learning algorithms on geospatial datasets for advancing research on climate modeling for energy systems. 
%I received my Masters in Civil Engineering from Penn State, focusing on hydrologic modeling and remote sensing.
%I'm currently an R{\&}D Associate at ORNL creating data science applications for national security. 
%I have spent time at Virginia Tech and Boeing Research {\&} Technology leading machine learning projects for a variety of applications.
%I'm looking for a position that leverages my knowledge of energy systems, data analytics, and environmental modeling.
%}

%----------------------------------------------------------------------------------------
%	Research Interest
%----------------------------------------------------------------------------------------
\cvsection{Research Interests}
\cvresearchinterest{Data analytics applications in engineering systems, machine learning pipelines, environmental modeling, spatial statistics, reinforcement/meta/few-shot learning, environmental remote sensing, geospatial modeling}

%----------------------------------------------------------------------------------------
%	EDUCATION
%----------------------------------------------------------------------------------------

\cvsection{Education}
\begin{cveducations}
  
  \cveducation{
    \begin{cveducationitems2}
         2017 \\ 
	\end{cveducationitems2}
    } % Year
    {{\bf University of Tennessee}} % Institution
    {
    \begin{cveducationitems} % Description(s)
        PhD in Energy Science \& Engineering \\
        Energy Science \& Engineering PhD Fellowship\\
	\end{cveducationitems}
    }
    
  \cveducation{2013} % Year
    {{\bf Pennsylvania State University -- University Park}} % Institution
    {
    \begin{cveducationitems} % Description(s)
        MS in Civil Engineering %\hfill Advisor: Dr. Michael Gooseff (CU-Boulder) \\ Concentration: Remote Sensing, Machine Learning\\
	\end{cveducationitems}
    }
    
  \cveducation{2011} % Year
    {{\bf University of Alabama -- Huntsville}} % Institution
    {
    \begin{cveducationitems} % Description(s)
        BS in Remote Sensing \& GIS \\Minor: Mathematics\\
	\end{cveducationitems}
    }    
    
%      \cveducation{2020--Current} % Year
%    {{\bf University of Tennessee}} % Institution
%    {
%    \begin{cveducationitems} % Description(s)
%        Graduate Certificate
%	\end{cveducationitems}
%    }   
    
\end{cveducations}


%-------------------------------------------------------------------------------
%	RESEARCH EXPERIENCE
%-------------------------------------------------------------------------------

\cvsection{Experience}
\begin{cvresearches}
%------------------------------------------------
\begin{cvresearch}{R{\&}D Associate}{Aug. 2019 -- present}{Cyber \& Applied Data Analytics Division}{Oak Ridge National Laboratory}
 \item Developing data analytics techniques (e.g. anomaly detection, visualizations, machine learning, etc.) for energy systems.
 \item Machine learning algorithms for rapid development of digital twins of industrial control systems.
 \item Writing research proposals/papers around machine learning applications for national security.
 \item Contributing to technology for over-the-air machine learning applications for secure communications. 
\end{cvresearch}
%------------------------------------------------
\begin{cvresearch}{Research Scientist}{Jan. 2019 -- Aug. 2019}{Hume Center for National Security \& Technology}{Virginia Tech}
 \item Developed new approaches for signal classification using deep learning methods. 
 \item Developed siamese networks for better classification of signals with high signal-to-noise ratio.
 \item Developed a Generative Adversarial Network modeling framework of latent space for extracting wanted features of synthetic images. 
 \item Contributed to Bayesian networks for understanding the likelihoods of known causes and contributing factors. 
% \item Publication: \cite{Langford:2019:RSC:3324921.3328781}
\end{cvresearch}

%------------------------------------------------
\begin{cvresearch}{Advanced Technologist}{Aug. 2016 -- Jan. 2019}{Boeing Research \& Technology}{Boeing}
 \item Lead projects for geospatial machine learning applications for aerospace industry.
  \item Contributed to object detection approaches to benefit the aerospace industry.
  \item Developed anomaly detection approach for identifying manufacturing defects.
 \item Hyperspectral remote sensing UAV sensor modeling for methane detection.
 %\item Multi/Hyperspectral image classification using deep learning techniques. 
 %\item Geospatial modeling (raster, vector, LiDAR processing) for inputs into autonomous aircraft taxing. 
\end{cvresearch}

\end{cvresearches}

\pagebreak

%------------------------------------------------
\cvsection{Research Experience}
\begin{cvresearches}

\begin{cvresearch}{Graduate Research Assistant}{Jan. 2014 -- Aug. 2016}{Bredesen Center}{University of Tennessee}
 %\item Working with Dr. Jitendra Kumar and Dr. Forrest Hoffman in the Climate Change Science Institute at ORNL.
 \item Provided geospatial datasets of Arctic ecosystems for parameterizing land surface models for climate change research.
 \item  Developed multi-sensor fusion frameworks of satellite imagery and ground measurements for retrieving surface parameters.
 \item  Developed automated algorithms for identifying and understanding disturbance threats
(e.g., wildfires) using spatiotemporal datasets in Google Earth Engine.
\end{cvresearch}

\begin{cvresearch}{ASTRO Internship}{Jan. 2014 -- Aug. 2014}{Computational Earth Sciences Group}{Oak Ridge National Laboratory}
 %\item Program offering short-term research opportunities with ORNL to current/recent graduate students. 
 \item Developed geospatial algorithms on high-resolution satellite imagery (i.e. 20 cm) for upscaling ground-based measurements.
 \item Continued research while enrolling into UTK/ORNL PhD program.
\end{cvresearch}

\begin{cvresearch}{Graduate Research Assistant}{2011 -- 2013}{Department of Civil and Environmental Engineering}{Pennsylvania State University}
 \item Utilized high-resolution, 50 cm -- 2 m, panchromatic \& multispectral satellite imagery
(QuickBird, WorldView-2, and GeoEye) to characterize soil moisture profiles in the McMurdo Dry Valleys, Antarctica using an object-based image classification. 
 \item Create a feed-forward neural network to estimate soil moisture values using multispectral satellite imagery and ground-based measurements.
\end{cvresearch}

%\begin{cvresearch}{Undergraduate Research Assistant}{2010 -- 2011}{Department of Atmospheric Sciences}{University of Alabama in Huntsville}
%\item Collaborated with NASA's Regional Visualization and Monitoring System (SERVIR) and
%Smithsonian Tropical Research Institute (STRI) to integrate satellite and geospatial data
%to support environmental monitoring in Central America.
%\end{cvresearch}

\end{cvresearches}

%----------------------------------------------------------------------------------------
% TECHNICAL STRENGTHS
%----------------------------------------------------------------------------------------
%% usage: \cvtechnicalstrength{<Type>}{<List>} 
\cvsection{Technical Strengths}
\begin{cvtechnicalstrengths}
  \cvtechnicalstrength
  {Languages} % Type
  {Python, MATLAB, C/C++, CUDA, MPI} % List
  \cvtechnicalstrength
  {Python Libraries} % Type
  {TensorFlow, PyTorch, MXNet, Scikit-learn, GeoPandas} % List
   \cvtechnicalstrength
  {Other Tools} % Type
  {LaTeX, SQL, Elasticsearch, DIRSIG, GRASS GIS, QGIS, ArcGIS} % List 
\end{cvtechnicalstrengths}

%----------------------------------------------------------------------------------------
%	PUBLICATION
%----------------------------------------------------------------------------------------


\cvsection{Selected Publications/Patents}
Full list at \href{https://scholar.google.com/citations?hl=en&user=8XedxuAAAAAJ&view_op=list_works&sortby=pubdate}{Google Scholar}


\begin{cvpublications}
\begin{bibunit}[tidycv]
  \nocite{Langford:2019:RSC:3324921.3328781}
  \nocite{Langford_DMESS2019_20191108}
  \nocite{Langford_RemoteSens_20190102}
   %\nocite{Langford-Dissertation_2017}
  \nocite{langford_20190158805}
   \nocite{Langford_RemoteSens_20160906}
  %\nocite{Langford-Theses_2013}
  \putbib[zlangford]
\end{bibunit}
\end{cvpublications}






\end{document}
